\documentclass[11pt,t]{beamer}
%%%%%%%%%%%%%%%%%%%%%%%%%%%%%%
% Packages
%%%%%%%%%%%%%%%%%%%%%%%%%%%%%%

\usepackage{geometry}              		 % 
\usepackage[english]{babel}              % Voor nederlandstalige hyphenatie (woordsplitsing)
\usepackage{amsmath,amsthm}              % Uitgebreide wiskundige mogelijkheden
\usepackage{url}                         % Om url's te verwerken
\usepackage{graphicx,subfigure}          % Om figuren te kunnen verwerken
\usepackage{color}						 % Om kleuren in Inkscape figuren te kunnen weergeven
\usepackage[utf8]{inputenc}              % Om niet ascii karakters rechtstreeks te kunnen typen
\usepackage{float}                       % Om nieuwe float environments aan te maken. Ook optie H!
\usepackage[section]{placeins}			 % Om ervoor te zorgen dat floats binnen dezelfde section blijven
\usepackage{eurosym}                     % om het euro symbool te krijgen
\usepackage{textcomp}                    % Voor onder andere graden celsius
%\usepackage{fancyhdr}                    % Voor fancy headers en footers
\usepackage{parskip}                     % Om paragrafen met een verticale spatie ipv horizontaal te laten beginnen
\usepackage{multicol}
%\usepackage[plainpages=false]{hyperref}  % Om hyperlinks te hebben in het pdfdocument



%%%%%%%%%%%%%%%%%%%%%%%%%%%%%%
% Layout
%%%%%%%%%%%%%%%%%%%%%%%%%%%%%%


%%%%%%%%%%%%%%%%%%%%%%%%%%%%%%
% Omgevingen
%%%%%%%%%%%%%%%%%%%%%%%%%%%%%%


%%%%%%%%%%%%%%%%%%%%%%%%%%%%%%
% Nieuwe commandos
%%%%%%%%%%%%%%%%%%%%%%%%%%%%%%

% De differentiaal operator
\newcommand{\diff}{\ensuremath{\mathrm{d}}}
\newcommand{\subsdiff}{\ensuremath{\mathrm{D}}}
\newcommand{\vardiff}{\ensuremath{\mathrm{\delta}}}

% Vectoren en matrices
\renewcommand{\vec}[1]{\ensuremath{\boldsymbol{#1}}}   % vector in bold
\newcommand{\mat}[1]{\ensuremath{\mathsf{#1}}}	    % matrix in serif font


% Graden celcius
\newcommand{\degC}{\ensuremath{^\circ \mathrm{C}}}

% nieuw commando om svg files dynamisch te updaten
\newcommand{\executeiffilenewer}[3]{%
\ifnum\pdfstrcmp{\pdffilemoddate{#1}}%
{\pdffilemoddate{#2}}>0%
{\immediate\write18{#3}}\fi%
}

% nieuw commando om svg files in te voeren en dynamisch te updaten
\newcommand{\includesvg}[2][0]{%
\executeiffilenewer{#2.svg}{#2.pdf}%
{inkscape -z -C --file=#2.svg %
--export-pdf=#2.pdf --export-latex}%
\ifx#10
	\let\svgwidth\undefined
\else
	\def\svgwidth{#1}
\fi%
\input{#2.pdf_tex}%
\ifx \svgwidth\undefined
\else
	\let\svgwidth\undefined
\fi%
}

% nieuw commando om fig files in te voeren
\newcommand{\includefig}[2][0]{%
\ifx#10
	\let\matwidth\undefined
\else
	\def\matwidth{#1}
\fi%
\input{#2.pdf_tex}%
\ifx \matwidth\undefined
\else
	\let\matwidth\undefined
\fi%
}

\mode<presentation> {\usetheme{kuleuven}}

%%%%%%%%%%%%%%%%%%%%%%%%%%%%%%%%%%%%%%%%
%info
\title{Title of the presentation}
\author{author 1, author 2, author 3}
\institute{Group, department, KU Leuven}
\subtitle{subtitle}
\date{January 6, 2013, Leuven, Belgium}
%pdf metadata
	\subject{KU Leuven's LaTeX template for oral presentation}
	\keywords{KU Leuven, LaTeX template, beamer, TikZ, pdfLaTeX}
\graphicspath{{graphics/}} % path to the graphics folder
%%%%%%%%%%%%%%%%%%%%%%%%%%%%%%%%%%%%%%%%
\begin{document}
%title page
	{
	\setbeamertemplate{footline}{}
	\begin{frame}
		\titlepage
	\end{frame}
	}
%%%%%%%%%%%%%%%%%%%%%%%%%%%%%%%%%%%%%%%%%%
\begin{frame}{Outline}
	\vskip 5mm
	\hfill	{\large \parbox{.95\textwidth}{\tableofcontents[hideallsubsections]}}
\end{frame}

%%%%%%%%%%%%%%%%%%%%%%%%%%%%%%%%%%%%%%%%
\section{Slides with text only}
%--------------------------------------
\begin{frame}{Itemize and enumerate ideas with overlays}
	\begin{itemize}
		\item \textcolor{black}{first item}
			\begin{itemize}
				\item first subitem
				\item second subitem
			\end{itemize}
		\item \textcolor{black}{second item}
			\begin{itemize}
				\item first subitem
				\item second subitem
			\end{itemize}
	\end{itemize}
\end{frame}
%--------------------------------------
\begin{frame}[t]{Double--column slide}
	\begin{columns}[t]
		\begin{column}{.5\textwidth}
			This is the top of the first column.	
		\end{column}
		\begin{column}{.5\textwidth}
			This is the top of the second column.
		\end{column}
	\end{columns}	
\end{frame}
%--------------------------------------
\begin{frame}{Fancy text and equations}
	\begin{center}~
		\begin{beamercolorbox}[wd=0.4\textwidth,rounded=true,center]{text}
			This is a beamer color box
		\end{beamercolorbox}
	\end{center}
	\begin{theorem}
		There exists an infinite set.
	\end{theorem}
	\begin{proof}
		This follows from the axiom of infinity.
	\end{proof}
	\vspace{4mm}
	\begin{equation}
		\textbf{M} \ddot{\textbf{q}} + \boldsymbol{\mathsf{\Phi}}_{\textbf{q}}^T \boldsymbol\lambda = \textbf{Q}
	\end{equation}
\end{frame}
%%%%%%%%%%%%%%%%%%%%%%%%%%%%%%%%%%%%%%%
\section{Slides with graphics and animations}
%--------------------------------------
\begin{frame}{Graphics and animation side by side}
	\begin{figure}
		\begin{minipage}[b]{0.49\linewidth}
			\centering
			\includegraphics[width=.3\textwidth,natwidth=265,natheight=314]{Tux.png}
		\end{minipage}
		\hfill	
		\begin{minipage}[b]{0.5\linewidth}
			\centering
		\end{minipage}
	\end{figure}
\end{frame}
%%%%%%%%%%%%%%%%%%%%%%%%%%%%%%%%%%%%%%%
\section{Further reading}
%--------------------------------------
\begin{frame}[t]{Further reading}
	\begin{itemize}
		\item BEAMER class user guide
		\item TikZ and PGF manual
	\end{itemize}
\end{frame}
\end{document}